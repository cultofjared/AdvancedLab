\documentclass[justified, nobib]{tufte-handout}

\title{Data Analysis with Python}
\author{Jared Garst}
\date{\today} % remove title space

\newcommand{\ipythonTutorialLink}
  {http://ipython.readthedocs.org/en/stable/interactive/index.html}
\newcommand{\jupyterInfoLink}
  {http://jupyter-notebook.readthedocs.org/en/latest/notebook.html\#basic-workflow}
\newcommand{\anacondaLink}{https://store.continuum.io/cshop/anaconda/}
\newcommand{\email}{mailto:jgarst@ucdavis.edu}
\newcommand{\aboutUnicodeLink}
  {http://www.joelonsoftware.com/articles/Unicode.html}
\newcommand{\librariesLink}{http://www.scipy.org/index.html}
\newcommand{\customizeDirectoryLink}
  {http://stackoverflow.com/questions/15680463/change-ipython-working-directory}
\newcommand{\listTutorialLink}
  {https://docs.python.org/2/tutorial/datastructures.html}
\newcommand{\pintLink}{http://pint.readthedocs.org/en/0.6/}
\newcommand{\hgLink}{http://mercurial.selenic.com/}
\newcommand{\gitLink}{http://git-scm.com/}
\newcommand{\lambdaTutorialLink}
  {https://pythonconquerstheuniverse.wordpress.com/2011/08/29/lambda_tutorial/}
\newcommand{\functionalProgrammingLink}
  {http://www.ibm.com/developerworks/library/l-prog/}
\newcommand{\matplotlibGalleryLink}{http://matplotlib.org/gallery.html}

\newcommand{\myMarginNote}[3][-9px]{\marginnote{\centering{#2} \\ \vspace{#1}
\justify{#3}}}

\newcommand{\matplotlibGalleryNote}{\footnote{You will need the syntax for
      matplotlib. You can find everything with the help commands, or can get
      syntax and ideas from \href{\matplotlibGalleryLink}{example plots} that
      others have made.}}

\newcommand{\floatNote}{\footnote{ Float stands for floating
  point number', otherwise known as a decimal number.
  Though not usually relevant for data analysis, be aware that
  \ipythoninline{int} and \ipythoninline{float} have detailed differences in
  precision and speed.}}

\begin{document}
\maketitle
\bigskip

\noindent
Python is a high level scripting language. While there are many reasons to like
or dislike Python, we are introducing it to you because it is becoming the
lingua franca of science.
This means that increasingly you can expect other people to understand
your code, and for many of your problems to already be solved and documented in
Python. Briefly, some exceptional Python features are:

\begin{description}
\item[Beautiful Syntax] \hfill \\
  Python will make your life easier, by encouraging simple, clean, easy to read
  code.
\item[Interpreted] \hfill \\
  The code you write is the code that runs. Nothing will be changed, nothing
  will be checked, nothing will be optimised.
\item[No Types] \hfill \\
  There is no easy way to specify how your data is to be interpreted.
\item[Program Control Through Indentation] \hfill \\
  Other languages use curly brackets \ipythoninline{{}} to determine program
  flow, and tabs by convention for readability.
  Python conflates the two.
\item[Batteries Included] \hfill \\
  Python makes it easy to do common tasks, has an extensive set of additional
  libraries, and a vibrant community maintaining them.
\end{description}

\noindent
IPython is an interactive version of python, designed for a more exploratory
approach to programming and analysis.
\sout{IPython} Jupyter notebooks allow you to quickly visualize your analysis,
and provide an easy record of your process.
Like any tools, you can become expert in
\href{\jupyterInfoLink}{Jupyter notebooks} and
\href{\ipythonTutorialLink}{IPython} syntax, but you won't need to for this
course, and we won't cover those skills.

\smallskip
\noindent
There are two versions of Python that you will encounter in the wild, Python 2
and Python 3. Python 3 is a take-no-prisoners reimagining of the the
language.
The result is a smoother language, that breaks everything down to the print
statement in Python 2.
Most libraries support both Python 2 and 3, but most code is in Python 2.
This means the appropriate version will depend on how old your project is.
This introduction will use Python 2 --- but for this course there aren't huge
differences between the choices.

\section*{Setup and Startup}
If you have never set up a programming environment before, the easiest way to
start is to install \href{\anacondaLink}{Anaconda}, distributed by Continum
Analytics. This modified python environment will install the most common python
libraries. It is distributed for free in hopes of later selling packages that
speed up your python code, but you can simply use it as a convenient
installation tool. If you want to try a grittier, more controlled installation
of Python and its many friends, you can schedule some time with
\href{\email}{Jared}.

\smallskip
\noindent
Once you have completed the installation, run \textinline{ipython notebook} in
the terminal, or the Anaconda IPython Notebook executable.
A web browser will automatically start with your notebook.
Managing directories in notebooks is a little clumsy.
Either make sure to run IPython in the directory you want, or
\href{\customizeDirectoryLink}{change the default directory} to match your
workflow.

\smallskip
\noindent
There are a few flavors of help commands available in IPython. Familiarity with
the options allows you to reference the manuals, or explore specific parts of
the language.  \marginnote[25pt]{
\centering{
  help commands\\}
\input{help_commands_no_output}
} \\



\begin{description}[font=\tt, leftmargin=0cm]

  \item[\ipythoninline{help()}] is the native python help command.
  Type \ipythoninline{help()} into the first cell, and execute with
  shift-enter.
  This is a good place to spend time getting familiar with the language, but
  you can also ask it about specific objects.
  For example, run \ipythoninline{help(list)} to see some common list
  manipulations.

  \item[\ipythoninline{?}] is IPython's own help command, and it is often nicer
  to use than \ipythoninline{help}.
  Try running \ipythoninline{?} to see an introduction to IPython.
  Like help you can also ask more specific commands, run
  \ipythoninline{import numpy; numpy?} to see an overview of the exceptionally
  useful numpy module.
  If you need all the gory details of an object, \ipythoninline{numpy??} will
  give you the source code.

  \item[\textinline{<regex>}\ipythoninline{?}] allows you to use
  \ipythoninline{?} to search through an object.
  Try running \ipythoninline{numpy.*cos*}\ipythoninline{?} to see every
  function in numpy related to $\cos$.

  \item[\PY{o}{\PYZpc{}}\ipythoninline{quickref}] is a magic function', unique to IPython,
  that you can use to see a verbatum list of all special IPython commands,
  including the magic functions.

  \item[\textinline{<tab>}] will provide code completion.
  Type but do not run \ipythoninline{numpy.arc} followed by a
  \textinline{<tab>} to have IPython suggest a list of inverse trig functions.

  \item[\textinline{Control-m h}] Provides a list of hotkeys for IPython
  notebooks.
  If you spend a lot of time in the notebook, these will speed you up.

\end{description}

\pagebreak
\section*{First Steps}
\input{first_example}

\pagebreak

\section*{Python Gotchas}
Some of the ways that python makes your life easier can send your program off
the rails, if you don't understand what the computer is doing.

\newthought{TYPES} still exist in python, even if you can't enforce
them. Consider the following python code:
\begin{adjustwidth}{2.0em}{0pt}
\input{types}
\end{adjustwidth}

\noindent
In the underlying C code, an operation on two integers always returns an
integer, because floating point\floatNote{} \;math is much slower. In python,
you can check a type with \ipythoninline{type(object)}, and enforce a
type with \ipythoninline{assert} and \ipythoninline{isinstance}.
\begin{adjustwidth}{2.0em}{0pt}
\input{division}
\end{adjustwidth}
\ipythoninline{assert} will stop the program with an error if the next
statement is not true.
Python 3 is smarter about division, and will return 1.5 for both calculations.
In the likely event that you cannot use python 3, you can still
import the behavior by including
\begin{adjustwidth}{2.0em}{0pt}
\input{future_division}
\end{adjustwidth}
at the top of your file or notebook.

\newthought{COPYING OBJECTS} can be done in two distinct ways, shown abstractly
in Figure~\ref*{fig:copy}. The more common method is to copy the object
reference, so that it can be used in a different part of the code. The second
method is when you want to have two distinct copies, so that you can have both
an original version and a modified version. In the first case it is appropriate
to duplicate the pointer. In the second case it is necessary to duplicate the
object itself.


\smallskip
\begin{marginfigure}
    \vspace*{\fill}
    \centering
    \subfloat[Case 1]{\scalebox{1}{\definecolor{cffffff}{RGB}{255,255,255}


\begin{tikzpicture}[scale=0.7, y=0.80pt,x=0.80pt,yscale=-1, inner sep=0pt, outer sep=0pt]
\begin{scope}[shift={(-1.0,-0.5)}]
  \path[draw=black,fill=cffffff] (101.0000,41.0000) ellipse (1.1289cm and
    1.1289cm);
  \path[draw=black,miter limit=10.00,line width=1.040pt] (175.4493,153.4493) --
    (137.0800,77.8500);
  \path[draw=black,miter limit=10.00,line width=1.040pt] (26.5507,153.4493) --
    (64.9200,77.8500);
  \path[rounded corners=0.0000cm] (1.0000,111.0000) rectangle (71.0000,131.0000);
  \begin{scope}[shift={(0.39716,153.06265)}]
  \end{scope}
  \path[rounded corners=0.0000cm] (131.0000,111.0000) rectangle
    (201.0000,131.0000);
  \begin{scope}[shift={(150.04249,153.06265)}]
  \end{scope}
  \path[rounded corners=0.0000cm] (81.0000,21.0000) rectangle (121.0000,61.0000);
  \begin{scope}[shift={(83.0,26.0)}]
  \end{scope}
  \path[draw=black,fill=black,miter limit=10.00] (134.6900,73.1800) --
    (141.0000,77.8100) -- (137.0800,77.8500) -- (134.7700,81.0000) -- cycle;
  \path[draw=black,fill=black,miter limit=10.00] (67.3100,73.1800) --
    (61.0000,77.8100) -- (64.9200,77.8500) -- (67.2300,81.0000) -- cycle;
\end{scope}

\end{tikzpicture}
}}

    \vfill

    \subfloat[Case 2]{\scalebox{0.69}{\definecolor{cffffff}{RGB}{255,255,255}

\begin{tikzpicture}[y=0.80pt,x=0.80pt,yscale=-1, inner sep=0pt, outer sep=0pt]
\begin{scope}[shift={(-12.28125,-0.5)}]
  \path[shift={(47.03197,0)},draw=black,fill=cffffff] (41.0000,41.0000) ellipse
    (1.1289cm and 1.1289cm);
  \path[draw=black,miter limit=10.00,line width=1.040pt] (51.9520,77.8500) --
    (12.8520,154.1800);
  \path[draw=black,fill=black,miter limit=10.00] (54.3420,73.1800) --
    (48.0320,77.8100) -- (51.9520,77.8500) -- (54.2620,81.0000) -- cycle;
  \path[rounded corners=0.0000cm] (68.0320,21.0000) rectangle (108.0320,61.0000);
  \path[draw=black,fill=cffffff] (191.0000,41.0000) ellipse (1.1289cm and
    1.1289cm);
  \path[draw=black,miter limit=10.00,line width=1.040pt] (266.1800,154.1800) --
    (227.0800,77.8500);
  \path[draw=black,fill=black,miter limit=10.00] (224.6900,73.1800) --
    (231.0000,77.8100) -- (227.0800,77.8500) -- (224.7700,81.0000) -- cycle;
  \path[rounded corners=0.0000cm] (171.0000,21.0000) rectangle (211.0000,61.0000);
\end{scope}
\begin{scope}[shift={(-12.78125,-1.0)}]
\end{scope}

\end{tikzpicture}}}
    %\subcaption{Case 2}
  \caption{The finger pointing at the moon is not the moon}
\label{fig:copy}
\end{marginfigure}

\noindent
Because the first case is common and the second case is expensive, python
duplicates the reference by default.

\smallskip
\begin{adjustwidth}{2.0em}{0pt}
  \input{copying}
\end{adjustwidth}

\newthought{THE NOTEBOOK STATE} is changed every time you run a command, not
every time you add a cell. This allows you to create confusing code, by removing
cells or executing them in a strange order.

\begin{adjustwidth}{2.5em}{0pt}
\input{notebook_nonlinear}
\end{adjustwidth}

\noindent
You can debug this problem by paying attention to the numbers in front of the
cells. Commands are executed in sequence with their numbers, not their
position. Try to keep your notebook linear, by not changing cells that have
already successfully executed.

\pagebreak

\section*{Problems}
\vspace{-0.5cm}

\begin{itemize}
\item[] \newthought{{\Large 1} \hspace{0.25em} DOCUMENTATION FROM PYTHON} \\
  Use the help commands to locate the \ipythoninline{scipy.stats} documentation
  on the Poisson distribution.
  Is the distribution a function or an object?
  Plot\matplotlibGalleryNote{} the expected number of counts for 10 trials of
  a poisson process, each with an average of 3 counts per trial.

\item[] \newthought{{\Large 2} \hspace{0.25em} DOCUMENTATION FROM GOOGLE} \\
  Try calling the poisson distribution function from
  \ipythoninline{scipy.stats} with no arguments.
  You will get an error message saying one argument was passed, but two were
  needed.
  Use google to explain these strange parameter counts.

\item[] \newthought{{\Large 3} \hspace{0.25em} USING PYTHON TO CALCULATE} \\
  Reproduce the mean, variance, and standard deviation of the data using your
  own functions.

  % \item The $\chi^2$ for a frequency distribution is traditionally taken to
  %   be $$ \mathlarger{\mathlarger{\sum}} \; \frac{\left( \text{observed} -
  %     \text{expected}\right)^2}{\text{expected}}$$

  %   How does this compare to a $\chi^2$ test with the points have errors
  %   associated with them? \\
  %   What distribution would you use to make the two equivalent?

%\item

\item[]
\newthought{{\Large 4} \hspace{0.25em} OBSERVE THE CENTRAL LIMIT THEOREM} \\
  Write a function \ipythoninline{data_avg(list, int)} that returns a new list
  $\bar{x}_n$, containing all possible averages of n values selected from the
  list.
  Use this to plot $\bar{x}_2$, $\bar{x}_{10}$ and $\bar{x}_{100}$.
  What are the means of these averaged distributions?
  What are the variances?
  How do these compare to the mean and variance of $x$?
\end{itemize}

\pagebreak

\section*{More Reading}
\begin{description}
\item \href{\listTutorialLink}{Data Structures and List Maniupulation}

\item \href{\pintLink}{Units and Errors}

\item Source control: \href{\hgLink}{Mercurial} and \href{\gitLink}{Git}

%\item \href{\lambdaTutorialLink}{Lambda Functions} and
%  \href{\functionalProgrammingLink}{Functional programming}

\end{description}
\end{document}
